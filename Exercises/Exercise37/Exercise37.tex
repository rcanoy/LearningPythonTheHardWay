\documentclass[a4paper, 12pt]{article}
\usepackage[utf8]{inputenc}
\usepackage[english]{babel}
\usepackage[margin=1in]{geometry}
\usepackage{tcolorbox}
\usepackage{listings}
\usepackage{color, soul}

\begin{document}
\title{\textbf{Exercise 37} \\ \textbf{Symbol Review}}
\author{Raymart Jay E. Canoy}
\date{\today}
\maketitle

\section{Keywords}
\begin{itemize}
\item{del: Used to delete objects, variables, lists or part of a list.}
\begin{tcolorbox}
\begin{lstlisting}
# defining a class
class myClass:
	name = "John"

# defining a variable
x = "Hello"

# defining a list
y = ["apple", "banana", "cherry"]

del myClass, x, y
print(myClass, x, y)
\end{lstlisting}
\end{tcolorbox}
\item{from: Used to import only a specified section from a module}
\item{as: Used to create a alias}
\item{global: Declares a global variable inside a function, and use it outside the function}
\begin{tcolorbox}
\begin{lstlisting}
def myFunc():
	global z
	z = "This is a global variable"
	
myFunc()
print(z)
\end{lstlisting}
\end{tcolorbox}
\item{with: Used in exception handling to make the code cleaner and much more readable. It simplifies the management of common resources like file streams.}
\pagebreak
\begin{tcolorbox}
\begin{lstlisting}
# 1) without using the with statement
file = open("file_path", "w")
file.write("hello world!")
file.close()

# 2) without using with statement
file = open("file_path", 'w')
try:
	file.write("Hello World!")
finally:
	file.close()
	
# 3) Using the with statement
with open("file_path", 'w') as file:
	file.write("Hello World!")
\end{lstlisting}
\end{tcolorbox}
\item{assert: Used when debugging code. Lets you test if a condition in your code returns True, if not, the program will raise an AssertionError.}
\item{pass: Used as a placeholder for future code. When pass statement is executed, nothing happens, but you avoid getting an error when empty code is not allowed.}
\begin{tcolorbox}
\begin{lstlisting}
def myFunc():
	pass
\end{lstlisting}
\end{tcolorbox}
\item{yield: similar to a return statement used for returning values in Python which returns a generator object to the one who calls the function which contains yield, instead simply returning the value.}
\begin{tcolorbox}
\begin{lstlisting}
def find_even(list_):
	for elem_ in list_:
		if elem_ % 2 == 0:
			yield elem_
			
list_ = [i for i in range(101)]
list_even = []

for j in find_even(list_):
	list_even.append(j)
\end{lstlisting}
\end{tcolorbox}

\pagebreak

\begin{tcolorbox}
\begin{lstlisting}
def find_word(list_, word_):
	for word in list_:
		if word == word_:
			yield word
			
list_ = "Geeks are gorgeous"
count = 0

for count_ in find_word(list_.lower()split(), "geeks":
	count += 1
	
print(count)
\end{lstlisting}
\end{tcolorbox}
\item{break: Used to break out a for loop, or a while loop.}
\item{try: Used in try...except blocks. Defines a block of code test if it contains any errors.}
\item{except: A keyword used in the try...except blocks. It defines a block of code to run if the try block raises an error.}
\item{finally: Used in try...except...else blocks if the block is final. This block will be executed no matter if the try block raises an error or not.}
\item{exec: Used for the dynamic execution of Python programs which can either be a string or object code.}
\begin{tcolorbox}
\begin{lstlisting}
exec(object[, globals[, locals]])
>>> exec("print('Add: %d' % (5+1))")
>>> exec("print(dir())", {})
>>> exec("print(dir())", {'dir':dir, 'fact':factorial})
\end{lstlisting}
\end{tcolorbox}
\item{raise: Used to raise an exception.}
\begin{tcolorbox}
\begin{lstlisting}
x = "Hello"

if not type is int:
	raise TypeError("Only integers are allowed.")
\end{lstlisting}
\end{tcolorbox}
\item{continue}
\item{is}
\item{lambda}
\end{itemize}
\end{document}